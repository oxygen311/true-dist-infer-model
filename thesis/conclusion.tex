\startconclusionpage

В рамках выпускной квалификационной работы был предложен новый метод оценки истинного эволюционного расстояния между геномами в рамках модели, предложенной в \cite{fr-4}.
Показана высокая точность данного метода.

Новый метод базируется на асимптотическом анализе комбинаторных формул для среднего числа компонент в общем случае.
Все формулы выведены в общем виде для любых весов $p_i$ на рёбрах, при этом существенно использовалась формула Кэли \cite{cayley}.
Для получения конечных результатов, вычисляется многомерный интеграл по плотности вероятности для стационарного распределения рассматриваемой модели.
При необходимости данные формулы могут быть применены и к другим распределениям весов на рёбрах.

В работе произведено сравнение эмпирических и теоретических результатов данной модели.
Показана их высокая согласованность.

Также выполнено сравнение нового метода оценки с методом, предложенным в \cite{fr-4}. 
Показано, что новый метод дает более высокие точность, применимость и эффективность.

Разные методы оценки эволюционного расстояния были применены к реальным  данным геномов из семейства \emph{Rosaceae}, класса \emph{Mammalian} и рода \emph{Shigella}.
Показано, что истинное эволюционное может существенно отличаться от минимального, уточнена граница парсимонии.