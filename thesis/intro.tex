%% Макрос для введения. Совместим со старым стилевиком.
\startprefacepage

Для многих филогенетических исследований, изучающих эволюционные связи, является важной возможность оценивать эволюционное расстояние между различными видами.
Самое первое решение подобной задачи заключалось в оценке минимального расстояния, которое требуется для преобразования одного генома в другой.
Предположение о том, что для преобразования одного генома в другой было сделано минимальное число перестроек, называется предположением парсимонии.

Однако в реальном процессе эволюции данное предположение может не быть выполнено, поэтому необходимо иметь оценку, которая не будет опираться на предположение парсимонии.
Подобную оценку, которая оценивает реальное расстояние между двумя геномами, а не минимальное, принято называть истинным эволюционным расстоянием.
Данный термин впервые был предложен в \cite{termin}.

На данный момент уже существует несколько методов оценки эволюционнного расстояния.
В \cite{alexeev-1} показано, что истинное количество геномных перестроек между некоторыми видами дрожжей отличается от минимально возможного на $~20\%$, а также разработан метод для оценки этого истинного расстояния.
При этом в работе существенно использовали модель случайных графов Эрдеша-Реньи \cite{erdos}.
В статье \cite{fr-4} была высказана критика данной модели. В ней все геномные перестройки происходят равновероятно.
Но в реальной жизни некоторые регионы имеют больший шанс быть вовлеченными в перестройку, а некоторые меньшую.

В данной работе рассмотрена новая модель генома, предложенная в \cite{fr-4}.
Эта модель учитывает факт того, что разные регионы генома подвержены перестройкам в разной степени.
Также эта модель является <<хрупкой>> ~--- это означает, что только определенные <<хрупкие>> геномные области подвержены перестройкам.
Проведён эмпирический и теоретический анализ данной модели.
Построен алгоритм оценивания истинного эволюционного расстояния и проведено его сравнение с другими подходами.